\documentclass[11pt,a4paper]{article}
\usepackage[utf8]{inputenc}
\usepackage[english]{babel}
\usepackage[left=2cm,right=2cm,top=3cm,bottom=4cm]{geometry}
\usepackage{amsmath}
\usepackage{mathtools}
\usepackage{cases}
\usepackage{hyperref}


\author{Ari-Pekka Honkanen}
\title{TBCALC: The Technical Document\\Version 1.0}
\begin{document}
\maketitle
\section{Introduction}
This documentation describes briefly the technical details and theoretical basis of $\textsc{tbcalc}$ package used to calculate the X-ray diffraction curves of toroidally bent, Johann-type crystal analysers. For comprehensive explanation, please refer to \cite{honkanen2020}.

\section{Calculation of the reflectivity curves}
As formally shown \cite{Honkanen_2016}, the effect of a constant component in a strain field to the diffraction curve can be taken into account by a applying a shift, either in energy or angle domain, to the Takagi-Taupin curve calculated without it. Since for toroidally bent crystal analysers the total strain field can be divided into a sum of depth-dependent and transversally varying parts, this allows efficient calculation of the reflectivity curves even for very large wafers. The calculation is summed up in the following steps:
\begin{itemize}
\item Compute the 1D Takagi-Taupin curve for the depth-dependent component of the strain field. \textsc{tbcalc} uses another Python package \textsc{pyTTE} for this.
\item Calculate distribution the energy or angle shifts due to the transversally varying component. The Johann error can be included in this part.
\item Convolve the 1D TT-curve with the shift distribution to obtain the full reflectivity curve of the analyser.
\item Convolve the result with the incident bandwidth, if needed.
\end{itemize}

\subsection{Depth-dependent Takagi-Taupin curve}
The 1D TT-curve is calculated using \textsc{PyTTE}. In v. 1.0 of \textsc{tbcalc} it is assumed that the main axes of curvature of TBCA:s are along the meridional and sagittal directions with respect to the diffraction plane and coincide, respectively, with the $x$- and $y$-axes of the Cartesian system used in the code and the manuscript \cite{honkanen2020}. By default, the internal anisotropic compliance matrices\footnote{Values from CRC Handbook of Chemistry and Physics, 82nd edition (2001)} are used for elastic parameters and \textsc{xraylib}\footnote{\url{https://github.com/tschoonj/xraylib}} for crystallograpic parameters and structure factors.

\subsection{Transverse stress and strain tensor fields}
For convenience, this section lists the equations for the transverse stress tensor and the strain it causes. Refer to \cite{honkanen2020} for derivation and discussion.
\subsubsection{Isotropic circular}
The components of the transverse stress tensor of an isotropic circular wafer with the diameter $L$ and meridional and sagittal bending radii $R_1$ and $R_2$, respectively, are
\begin{equation}
\sigma_{xx} =  \frac{E}{16 R_1 R_2}\left( \frac{L^2}{4} - x^2 - 3 y^2 \right) \qquad
\sigma_{xy} = \frac{E}{8 R_1 R_2}x y \qquad
\sigma_{yy} =  \frac{E}{16 R_1 R_2}\left(\frac{L^2}{4} - 3 x^2 - y^2\right)
\end{equation}
the corresponding strain tensor components
\begin{gather}
u_{xx} = \frac{1}{16 R^2}\left[(1-\nu)\frac{L^2}{4} - (1-3\nu)x^2 - (3-\nu)y^2 \right] \\
u_{yy} = \frac{1}{16 R^2}\left[(1-\nu)\frac{L^2}{4} - (1-3\nu)y^2 - (3-\nu)x^2 \right] \\
u_{xy} = \frac{1+\nu}{8 R^2}x y \qquad u_{xz} = u_{yz} = 0 \qquad
u_{zz} = \frac{\nu}{4R^2}\left(x^2 + y^2 -\frac{L^2}{8} \right)
\end{gather}
and the contact force per unit area
\begin{equation}
P = \frac{E d}{16 R_1^2 R_2^2} \left[ 
\left(3 R_1 + R_2 \right) x^2
+ \left(R_1 + 3 R_2 \right) y^2
- \left(R_1 + R_2 \right)\frac{L^2}{4}
\right].
\end{equation}

\bibliographystyle{unsrt}
\bibliography{documentation}
\end{document}