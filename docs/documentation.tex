\documentclass[11pt,a4paper]{article}
\usepackage[utf8]{inputenc}
\usepackage[english]{babel}
\usepackage[left=2cm,right=2cm,top=3cm,bottom=4cm]{geometry}
\usepackage{amsmath}
\usepackage{mathtools}
\usepackage{cases}
\usepackage{hyperref}


\author{Ari-Pekka Honkanen}
\title{TBCALC: The Technical Document\\Version 1.0}
\begin{document}
\maketitle
\section{Introduction}
This documentation describes briefly the technical details and theoretical basis of $\textsc{tbcalc}$ package used to calculate the X-ray diffraction curves of toroidally bent, Johann-type crystal analysers. For comprehensive explanation, please refer to \cite{honkanen2020}.

\section{Calculation of the reflectivity curves}
As formally shown \cite{Honkanen_2016}, the effect of a constant component in a strain field to the diffraction curve can be taken into account by a applying a shift, either in energy or angle domain, to the Takagi-Taupin curve calculated without it. Since for toroidally bent crystal analysers the total strain field can be divided into a sum of depth-dependent and transversally varying parts, this allows efficient calculation of the reflectivity curves even for very large wafers. The calculation is summed up in the following steps:
\begin{itemize}
\item Compute the 1D Takagi-Taupin curve for the depth-dependent component of the strain field. \textsc{tbcalc} uses another Python package \textsc{pyTTE} for this.
\item Calculate distribution the energy or angle shifts due to the transversally varying component. The Johann error can be included in this part.
\item Convolve the 1D TT-curve with the shift distribution to obtain the full reflectivity curve of the analyser.
\item Convolve the result with the incident bandwidth, if needed.
\end{itemize}

\subsection{Depth-dependent Takagi-Taupin curve}
The 1D TT-curve is calculated using \textsc{PyTTE}. In v. 1.0 of \textsc{tbcalc} it is assumed that the main axes of curvature of TBCA:s are along the meridional and sagittal directions with respect to the diffraction plane and coincide, respectively, with the $x$- and $y$-axes of the Cartesian system used in the code and the manuscript \cite{honkanen2020}. By default, the internal anisotropic compliance matrices\footnote{Values from CRC Handbook of Chemistry and Physics, 82nd edition (2001)} are used for elastic parameters and \textsc{xraylib}\footnote{\url{https://github.com/tschoonj/xraylib}} for crystallograpic parameters and structure factors.

\bibliographystyle{unsrt}
\bibliography{documentation}
\end{document}