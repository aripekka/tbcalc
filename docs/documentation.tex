\documentclass[11pt,a4paper]{article}
\usepackage[utf8]{inputenc}
\usepackage[english]{babel}
\usepackage[left=2cm,right=2cm,top=3cm,bottom=4cm]{geometry}
\usepackage{amsmath}
\usepackage{mathtools}
\usepackage{cases}
\usepackage{hyperref}


\author{Ari-Pekka Honkanen}
\title{TBCALC: The Technical Document\\Version 1.0}
\begin{document}
\maketitle
\section{Introduction}
This documentation describes briefly the technical details and theoretical basis of $\textsc{tbcalc}$ package used to calculate the X-ray diffraction curves of toroidally bent, Johann-type crystal analysers. For comprehensive explanation, please refer to \cite{honkanen2020}.

\section{Calculation of the reflectivity curves}
As formally shown \cite{Honkanen_2016}, the effect of a constant component in a strain field to the diffraction curve can be taken into account by a applying a shift, either in energy or angle domain, to the Takagi-Taupin curve calculated without it. Since for toroidally bent crystal analysers the total strain field can be divided into a sum of depth-dependent and transversally varying parts, this allows efficient calculation of the reflectivity curves even for very large wafers. The calculation is summed up in the following steps:
\begin{itemize}
\item Compute the 1D Takagi-Taupin curve for the depth-dependent component of the strain field. \textsc{tbcalc} uses another Python package \textsc{pyTTE} for this.
\item Calculate distribution the energy or angle shifts due to the transversally varying component. The Johann error can be included in this part.
\item Convolve the 1D TT-curve with the shift distribution to obtain the full reflectivity curve of the analyser.
\item Convolve the result with the incident bandwidth, if needed.
\end{itemize}

\subsection{Depth-dependent Takagi-Taupin curve}
The 1D TT-curve is calculated using \textsc{PyTTE}. In v. 1.0 of \textsc{tbcalc} it is assumed that the main axes of curvature of TBCA:s are along the meridional and sagittal directions with respect to the diffraction plane and coincide, respectively, with the $x$- and $y$-axes of the Cartesian system used in the code and the manuscript \cite{honkanen2020}. By default, the internal anisotropic compliance matrices\footnote{Values from CRC Handbook of Chemistry and Physics, 82nd edition (2001)} are used for elastic parameters and \textsc{xraylib}\footnote{\url{https://github.com/tschoonj/xraylib}} for crystallograpic parameters and structure factors.

\subsection{Transverse stress and strain tensor fields \label{sec:deformations}}
For convenience, this section lists the equations for the transverse stress tensor and the strain it causes. Refer to \cite{honkanen2020} for the derivation and discussion.
\subsubsection{Isotropic circular}
The components of the transverse stress tensor of an isotropic circular wafer with the diameter $L$ and meridional and sagittal bending radii $R_1$ and $R_2$, respectively, are
\begin{equation}
\sigma_{xx} =  \frac{E}{16 R_1 R_2}\left( \frac{L^2}{4} - x^2 - 3 y^2 \right) \qquad
\sigma_{xy} = \frac{E}{8 R_1 R_2}x y \qquad
\sigma_{yy} =  \frac{E}{16 R_1 R_2}\left(\frac{L^2}{4} - 3 x^2 - y^2\right)
\end{equation}
the corresponding strain tensor components
\begin{gather}
u_{xx} = \frac{1}{16 R_1 R_2}\left[(1-\nu)\frac{L^2}{4} - (1-3\nu)x^2 - (3-\nu)y^2 \right] \\
u_{yy} = \frac{1}{16 R_1 R_2}\left[(1-\nu)\frac{L^2}{4} - (1-3\nu)y^2 - (3-\nu)x^2 \right] \\
u_{xy} = \frac{1+\nu}{8 R_1 R_2}x y \qquad u_{xz} = u_{yz} = 0 \qquad
u_{zz} = \frac{\nu}{4 R_1 R_2}\left(x^2 + y^2 -\frac{L^2}{8} \right)
\end{gather}
and the contact force per unit area
\begin{equation}
P = \frac{E d}{16 R_1^2 R_2^2} \left[ 
\left(3 R_1 + R_2 \right) x^2
+ \left(R_1 + 3 R_2 \right) y^2
- \left(R_1 + R_2 \right)\frac{L^2}{4}
\right].
\end{equation}
\subsubsection{Anisotropic circular}
The stretching stress tensor components are 
\begin{equation}
\sigma_{xx} =  \frac{E'}{16 R_1 R_2}\left( \frac{L^2}{4} - x^2 - 3 y^2 \right) \quad 
\sigma_{yy} =  \frac{E'}{16 R_1 R_2}\left(\frac{L^2}{4} - 3 x^2 - y^2\right) \quad
\sigma_{xy} = \frac{E'}{8 R_1 R_2}x y
\end{equation}
where 
\begin{equation}
E' = \frac{8}{3(S_{11}+S_{22})+2 S_{12}+S_{66}},
\end{equation}
the corresponding strain tensor 
\begin{align}
u_{xx} &= \frac{E'}{16 R_1 R_2} \left[ (S_{11}+S_{12})\frac{L^2}{4} - (S_{11} + 3 S_{12}) x^2  -(3 S_{11} + S_{12}) y^2 + 2 S_{16} xy \right] \\
u_{yy} &= \frac{E'}{16 R_1 R_2} \left[ (S_{21}+S_{22})\frac{L^2}{4} - (S_{21} + 3 S_{22}) x^2  -(3 S_{21} + S_{22}) y^2 + 2 S_{26} xy \right] \\
u_{zz} &= \frac{E'}{16 R_1 R_2} \left[ (S_{31}+S_{32})\frac{L^2}{4} - (S_{31} + 3 S_{32}) x^2  -(3 S_{31} + S_{32}) y^2 + 2 S_{36} xy \right] \\
u_{xz} &= \frac{E'}{32 R_1 R_2} \left[ (S_{41}+S_{42})\frac{L^2}{4} - (S_{41} + 3 S_{42}) x^2  -(3 S_{41} + S_{42}) y^2 + 2 S_{46} xy \right] \\
u_{yz} &= \frac{E'}{32 R_1 R_2} \left[ (S_{51}+S_{52})\frac{L^2}{4} - (S_{51} + 3 S_{52}) x^2  -(3 S_{51} + S_{52}) y^2 + 2 S_{56} xy \right] \\
u_{xy} &= \frac{E'}{32 R_1 R_2} \left[ (S_{61}+S_{62})\frac{L^2}{4} - (S_{61} + 3 S_{62}) x^2  -(3 S_{61} + S_{62}) y^2 + 2 S_{66} xy \right]
\end{align}
and the contact force per unit area
\begin{equation}
P = \frac{E' d}{16 R_1^2 R_2^2} \left[ 
\left(3 R_1 + R_2 \right) x^2
+ \left(R_1 + 3 R_2 \right) y^2
- \left(R_1 + R_2 \right)\frac{L^2}{4}
\right].
\end{equation}

\subsubsection{Isotropic rectangular}
The components of the transverse stress tensor of an isotropic rectangular wafer with the side lengths $a$ and $b$ aligned with the meridional and sagittal radii of curvature $R_1$ and $R_2$, respectively, are
\begin{align}
\sigma_{xx} &= \frac{E}{g R_1 R_2}\left[\frac{a^2}{12}- x^2 +\left(\frac{1+\nu}{2}+5\frac{ a^2}{b^2} +\frac{1-\nu}{2} \frac{a^4}{b^4}\right)\left(\frac{b^2}{12}-y^2 \right)\right]  \\
\sigma_{yy} &= \frac{E}{g R_1 R_2}\left[\frac{b^2}{12}- y^2 +\left(\frac{1+\nu}{2} +5\frac{ b^2}{a^2} +\frac{1-\nu}{2} \frac{b^4}{a^4}\right)\left(\frac{a^2}{12}-x^2 \right)\right]  \\
\sigma_{xy} &= \frac{2 E}{g R_1 R_2}xy, 
\end{align}
where
\begin{equation}
g = 8+10 \left(\frac{a^2}{b^2} + \frac{b^2}{a^2}\right) + (1-\nu)\left(\frac{a^2}{b^2} - \frac{b^2}{a^2}\right)^2.
\end{equation}
The stretching strain tensor components are
\begin{equation}
u_{xx} = \frac{\sigma_{xx}-\nu \sigma_{yy}}{E} \quad u_{yy} = \frac{\sigma_{yy}-\nu \sigma_{xx}}{E} \quad u_{xy} = \frac{1+\nu}{E}\sigma_{xy} \quad u_{xz}=u_{yz}=0 \quad  u_{zz} = -\frac{\nu}{E}(\sigma_{xx}+\sigma_{yy})
\end{equation}
and the contact force
\begin{align}
P = - \frac{Ed}{gR_1^2R_2^2}\Bigg[
&\left( R_1 \left(\frac{1+\nu}{2} +5\frac{ b^2}{a^2} +\frac{1-\nu}{2} \frac{b^4}{a^4}\right) + R_2\right)\left(\frac{a^2}{12}-x^2 \right) \nonumber \\ +
&\left( R_2 \left(\frac{1+\nu}{2}+5\frac{ a^2}{b^2} +\frac{1-\nu}{2} \frac{a^4}{b^4}\right)
+ R_1 \right)\left(\frac{b^2}{12}-y^2 \right) \Bigg]
\end{align}

\subsubsection{Anisotropic rectangular}
For an anisotropic rectangular wafer, the transverse stress tensor components are
\begin{align}
\sigma_{xx} &= C_{02} + 12 C_{22}x^2 + 24 C_{13} xy + 12 C_{04} y^2 \\
\sigma_{yy} &= C_{20} + 12 C_{22}y^2 + 24 C_{31} xy + 12 C_{40} x^2 \\
\sigma_{xy} &= -C_{11}  - 12 C_{31} x^2 - 24 C_{22} xy - 12 C_{13} y^2 
\end{align}
from which we can calculate the corresponding strain tensor
\begin{align}
u_{xx} &= S_{11} \sigma_{xx} + S_{12} \sigma_{yy} + S_{16} \sigma_{xy} \\
u_{yy} &= S_{21} \sigma_{xx} + S_{22} \sigma_{yy} + S_{26} \sigma_{xy} \\
u_{xy} &= \frac{1}{2}\left(S_{61}\sigma_{xx} + S_{62} \sigma_{yy} + S_{66} \sigma_{xy} \right) \\
u_{xz} &= \frac{1}{2}\left(S_{41}\sigma_{xx} + S_{42} \sigma_{yy} + S_{46} \sigma_{xy} \right) \\
u_{yz} &= \frac{1}{2}\left(S_{51}\sigma_{xx} + S_{52} \sigma_{yy} + S_{56} \sigma_{xy} \right)  \\
u_{zz} &= S_{31} \sigma_{xx} + S_{32} \sigma_{yy} + S_{36} \sigma_{xy}
\end{align}
and the contact force per surface area
\begin{equation}
P = -d \left(\frac{\sigma_{xx}}{R_1} + \frac{\sigma_{yy}}{R_2} \right).
\end{equation}
The coefficients $C_{ij}$ are obtained by solving the matrix equation $\Lambda  \mathbf{C} = \mathbf{b}$ in terms of $\mathbf{C}$ where
\begin{equation}
\mathbf{C} = \left[ \begin{matrix}
C_{11} & C_{20} & C_{02} & C_{22} & C_{31} & C_{13} & C_{40} & C_{04} & \lambda_1
\end{matrix}
\right]^{\mathrm{T}},
\end{equation}
\begin{equation}
\mathbf{b} = \left[ \begin{matrix}
0 & 0 & 0 & 0 & 0 &0 &0 & 0 & -(24 R_1 R_2)^{-1}
\end{matrix}
\right]^{\mathrm{T}},
\end{equation}
and
\begin{equation}
\Lambda = \left[
\begin{matrix}
 S_{66} & -S_{26} & -S_{16} & \Lambda_{14} & S_{66} a^2 & S_{66} b^2 & -S_{26} a^2 & -S_{16} b^2 & 0 \\
-S_{26} & S_{22} & S_{12} & \Lambda_{24}  & - S_{26} a^2 & -S_{26} b^2 & S_{22} a^2 & S_{12} b^2 & 0 \\
-S_{16} & S_{12} & S_{11} & \Lambda_{34} & - S_{16} a^2 & -S_{16} b^2 & S_{12} a^2 & S_{11} b^2 & 0 \\
 \Lambda_{41} & \Lambda_{42} & \Lambda_{43} & \Lambda_{44} & \Lambda_{45} & \Lambda_{46} & \Lambda_{47} & \Lambda_{48} & \Lambda_{49} \\
5 S_{66} a^2 & -5 S_{26} a^2 & -5 S_{16} a^2 & \Lambda_{54} & \Lambda_{55} & \Lambda_{56} & -9 S_{26} a^4 & -5 S_{16} a^2 b^2 & - 2 S_{26} \\
5 S_{66} b^2 & -5 S_{26} b^2 & -5 S_{16} b^2 & \Lambda_{64} & \Lambda_{65} & \Lambda_{66} & -5 S_{26} a^2 b^2 & -9 S_{16} b^4 & - 2 S_{16} \\
-5 S_{26} a^2 & 5 S_{22} a^2 & 5 S_{12} a^2 & \Lambda_{74} & -9 S_{26} a^4 & -5 S_{26} a^2 b^2 & 9 S_{22} a^4 & 5 S_{12} a^2 b^2 & S_{22} \\
-5 S_{16} b^2 & 5 S_{12} b^2 & 5 S_{11} b^2 & \Lambda_{84} & -5 S_{16} a^2 b^2 & -9 S_{16} b^4 & 5 S_{12} a^2 b^2 & 9 S_{11} b^4 & S_{11} \\
0 & 0 & 0 & \Lambda_{94} & -2 S_{26} & -2 S_{16} & S_{22} & S_{11} & 0
\end{matrix}
\right]
\end{equation}
with
\begin{align}
\Lambda_{14} &= -S_{16} a^2 -S_{26}b^2 & \Lambda_{24} &= S_{12}a^2 + S_{22}b^2 \nonumber  \\
\Lambda_{34} &= S_{11}a^2 + S_{12}b^2 & \Lambda_{41} &= -5 S_{16}a^2 - 5 S_{26}b^2 \nonumber  \\
\Lambda_{42} &= 5 S_{12}a^2 + 5 S_{22}b^2 & \Lambda_{43} &= 5 S_{11}a^2 + 5 S_{12}b^2 \nonumber \\
\Lambda_{44} &= 9 S_{11} a^4 + 9 S_{22} b^4 + 10 (S_{12}+2 S_{66})a^2 b^2 & 
\Lambda_{45} &= -9 S_{16} a^4 -25 S_{26} a^2 b^2 \nonumber  \\
\Lambda_{46} &= -25 S_{16}a^2b^2 -9 S_{26} b^4 & \Lambda_{47} &= 9 S_{12} a^4 + 5 S_{22} a^2 b^2 \nonumber \\
\Lambda_{48} &= 5 S_{11} a^2 b^2 + 9 S_{12} b^4 & \Lambda_{49} &= 2 S_{12} + S_{66} \nonumber \\
\Lambda_{54} &= -9 S_{16} a^4 -25 S_{26} a^2 b^2 & \Lambda_{55} &= 9 S_{66} a^4 + 20 S_{22} a^2 b^2 \nonumber \\
\Lambda_{56} &= 5(4 S_{12} + S_{66} ) a^2 b^2 & \Lambda_{64} &=  -25 S_{16} a^2 b^2 -9 S_{26} b^4 \nonumber  \\
\Lambda_{65} &= 5(4 S_{12} + S_{66}) a^2 b^2 & \Lambda_{66} &= 20 S_{11} a^2 b^2 + 9 S_{66} b^4 \nonumber \\
\Lambda_{74} &= 9 S_{12} a^4 + 5 S_{22} a^2 b^2 & \Lambda_{84} &= 5 S_{11} a^2 b^2 + 9 S_{12} b^4 \nonumber \\
\Lambda_{94} &=  2 S_{12} + S_{66} & & \nonumber 
\end{align}

\subsubsection{Tensors in cylindrical coordinates}
Internally \textsc{tbcalc} performs the computations in the Cartesian coordinates $(x,y,z)$ but especially with the circular analysers expressing the strain and stress tensors in the cylindrical system $(r,\phi,z)$ can be useful. The implemented coordinate transform \texttt{cartesian\_tensors\_to\_cylindrical} uses the following formula
\begin{align}
T'_{rr} &=   \cos^2 \phi T_{xx}  + 2 \sin \phi \cos \phi  T_{xy} +  \sin^2 \phi T_{yy} \\
T'_{r \phi} &=  -  \sin \phi \cos \phi T_{xx} + (\cos^2 \phi - \sin^2 \phi) T_{xy} + \sin \phi \cos \phi T_{yy} \\
T'_{\phi \phi} &= \sin^2 \phi T_{xx} - 2 \sin \phi \cos \phi  T_{xy} + \cos^2 \phi T_{yy} \\
T'_{r z} &= \cos \phi T_{xz} + \sin \phi T_{yz} \\
T'_{\phi z} &= - \sin \phi T_{xz} + \cos \phi T_{yz} \\
T'_{z z} &= T_{zz}. 
\end{align}
Strictly speaking $\phi$ is actually handled here as $r \phi$ in order to keep the physical unit of the coordinates and thus the dimensions of the transformed tensor components consistent with the Cartesian representation.

\subsection{Energy and angle shifts due to transverse deformation}
Locally the shape of the reflection curve is given by the 1D TT-curve but its position on the energy or angle scale is shifted by the transverse deformations given in Section~\ref{sec:deformations}. When the energy is scanned the shifts are given by
\begin{equation}
\frac{\Delta \mathcal{E}}{\mathcal{E}} = 
- u_{zz} \cos^2 \phi - 2 u_{xz} \sin \phi \cos \phi
- u_{xx} \sin^2 \phi 
+\left[\left(u_{zz} - u_{xx} \right) \sin \phi \cos \phi + 2 u_{x z} \sin^2 \phi \right] \cot \theta_B
\end{equation}
where $\theta_B$ is the Bragg angle and $\phi$ is the asymmetry angle. Similarly in the rocking angle scan the shifts in angle are given by
\begin{equation}
\Delta \theta = - \left( u_{zz} \cos^2 \phi + 2 u_{xz} \sin \phi \cos \phi
+ u_{xx} \sin^2 \phi \right) \tan \theta_B + \left(u_{zz} - u_{xx} \right) \sin \phi \cos \phi + 2 u_{x z} \sin^2 \phi.
\end{equation}

\subsection{Johann error}
The shifts in the angle due to Johann error are
\begin{equation}
\Delta \theta = \frac{x^2}{2 R_1^2} \cot \theta 
- \frac{(R_1 -R_2)(R_1 \sin^2 \theta - R_2)}{2 R_1 R_2 \sin \theta \cos \theta}y^2
\end{equation} 
and in terms of energy
\begin{equation}
\Delta \mathcal{E} = -\frac{x^2}{2 R_1^2} \mathcal{E} \cot^2 \theta
+ \frac{(R_1 -R_2)(R_1 \sin^2 \theta - R_2)}{2 R_1 R_2 \sin^2 \theta} \mathcal{E}  y^2.
\end{equation}
Note that the expression for $\Delta \theta$ is derived expanding $\sin x$ to the first order and thus ceases to be valid near $\theta = \pi/2$.



\bibliographystyle{unsrt}
\bibliography{documentation}
\end{document}